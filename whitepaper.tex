\documentclass[12pt]{article}

% Packages
\usepackage{times}
\usepackage{hyperref}
\usepackage{graphicx}
\usepackage{amsmath}
\usepackage{enumitem}
\usepackage{setspace}
\usepackage{geometry}
\usepackage{booktabs}
\usepackage{array}
\usepackage{tikz}
\geometry{margin=1in}

\setlength{\parskip}{0.8em}
\setlength{\parindent}{0pt}

\hypersetup{
    colorlinks=true,
    linkcolor=black,
    urlcolor=blue,
    citecolor=black
}

\title{\textbf{PoliScore:\\
A Blueprint for Evidence-Based Democracy}\\[0.5em]
\large Draft 0.5 --- \today}
\date{}

\begin{document}

\maketitle

\begin{center}
\textit{\footnotesize
PoliScore is currently an unfunded, nights-and-weekends project maintained by a single independent developer.
The methodologies, frameworks, and evaluation systems described in this white paper should be understood as an
\emph{aspirational blueprint} for a world looking for a better way to engage with politics. Only a subset of
these ideas are partially prototyped in code today, and many components are still speculative or in early research.
The features currently deployed on \texttt{poliscore.us} reflect earlier and simplified versions of this framework
and should be interpreted as experimental demonstrations---not as a complete realization of the system described
here. Building PoliScore into a robust, trustworthy, and globally extensible public good will require collaborators,
funding, and institutional partners who share the goal of moving beyond click-driven political narratives toward
rigorous, transparent evaluation of law. While I have taken the time to paint a vision, it is you the reader who must
ultimately invest in it.}
\end{center}

\begin{abstract}
PoliScore consists of two core systems:
(1) a non-partisan framework for evaluating the quality and expected societal impact of individual bills, and
(2) an impact aggregation and interpretation pipeline for translating those bill-level assessments into clear,
transparent measures of legislative performance.

At the bill level, PoliScore defines a seven-pillar framework for policy quality---problem clarity, evidence support,
implementation feasibility, economic sustainability, distributional impact, governance integrity, and systemic risk---
derived from foundational insights in political philosophy, welfare economics, institutional theory, and systems
engineering. These dimensions are connected to a sectoral impact model that scores legislation on its predicted effects
across major policy domains (such as healthcare, education, energy, and national defense) and an ``Overall Impact to
Society'' metric.

At the legislator level, PoliScore sketches a transparent aggregation pipeline from bill-level impact scores and
legislator--bill interactions (sponsorship, co-sponsorship, and roll-call votes) to Intended Societal Impact scores
and interpretable letter grades for each legislator. The same framework supports parameterized natural-language
summaries that could explain a legislator's record in terms accessible to non-expert voters.

PoliBench, a companion benchmark suite, is proposed as a way to operationalize the seven-pillar framework into
reproducible evaluation tasks that test an AI system's ability to carry out policy reasoning, foresee unintended
consequences, identify governance vulnerabilities, and reason about distributional and sectoral impacts directly from
legislative text. Although early experiments with PoliScore use large language models as evaluators, the underlying
methodology is \emph{model-agnostic}: in principle, the same process could be executed by teams of human analysts
following a shared rubric or by alternative AI systems.

This paper is written in explicit contrast to the present media ecosystem, in which divisive narratives, outrage
cycles, and engagement-optimized feeds dominate public attention. PoliScore is offered as an alternate organizing
principle: a world where public debate is anchored in the structural quality and expected societal impact of legislation,
rather than in partisan branding or isolated viral moments.

Today, PoliScore is an unfunded solo effort and an MIT-licensed open-source project still in its infancy. The systems
described here will only be realized with help: engineers, researchers, civic technologists, journalists, and funders
who are willing to participate in building and governing a transparent, non-partisan infrastructure for evaluating law.
The paper concludes with limitations, ethical considerations, discussion of web-integrated evidence retrieval, and a
direct call for collaboration and institutional support.
\end{abstract}

\newpage
\tableofcontents
\newpage

%%%%%%%%%%%%%%%%%%%%%%%%%%%%%%%%%%%%%%%%%%%%%%%%%%%
%%%%%%%%%%%%%%%   INTRODUCTION   %%%%%%%%%%%%%%%%%%
%%%%%%%%%%%%%%%%%%%%%%%%%%%%%%%%%%%%%%%%%%%%%%%%%%%

\section{Introduction}
\label{sec:intro}

Public policy shapes the daily lives of individuals and the long-term trajectory of societies. Yet despite its immense
influence, there is no widely accepted, non-partisan standard for evaluating the quality of legislation before it is
enacted, nor a unified methodology for summarizing the long-run legislative performance of elected officials in terms
ordinary voters can understand.

Instead, the dominant information environment is organized around something quite different: attention. News feeds,
social platforms, and partisan media outlets are structurally incentivized to amplify spectacle, conflict, and simplified
narratives that drive clicks and engagement. The question is rarely, ``Is this bill well-designed and societally
beneficial?'' It is more often, ``Does this story fit our tribe's narrative, and will people share it?''

In that environment, voters face an impossible informational task. Organized misinformation campaigns, partisan media
ecosystems, and attention-maximizing platforms collectively produce a high-noise landscape in which the underlying policy
record of legislators is difficult to discern. It is easy to surface a few symbolic votes or speeches; it is far harder
to understand the cumulative impact of years of legislative activity on real human outcomes.

At the same time, advances in artificial intelligence have made it technically feasible to read and reason over large
volumes of legislative text, related reports, and contextual evidence. Used carefully, AI could help bridge the gap
between raw policy artifacts and structured, interpretable assessments. Used naively, AI risks becoming a new kind of
opaque authority, or simply a ``magic wand'' that produces persuasive but ungrounded judgments that replicate the worst
dynamics of the current media ecosystem.

PoliScore is an attempt to imagine something better.

It is not, as of this writing, a fully realized system with institutional backing and a large engineering team. It is a
vision and an emerging prototype:

\begin{itemize}
    \item to define a rigorous, academically grounded theory of policy quality,
    \item to formalize a repeatable process for evaluating bills along that theory,
    \item to translate those evaluations into sectoral and overall impact scores,
    \item to aggregate impacts into transparent, interpretable legislator grades,
    \item and to provide benchmark tasks for testing the AI systems used along the way.
\end{itemize}

The long-term vision can be stated simply:

\begin{quote}
\textit{If we can show that we can evaluate bills in a non-partisan, well-grounded, and reproducible way, then we can
evaluate legislators by aggregating the impacts of their bill-level interactions.}
\end{quote}

PoliScore therefore has two tightly coupled layers:

\begin{enumerate}
    \item a \textbf{policy quality layer}, which evaluates the structure, evidence base, and predicted societal impact
          of individual bills; and
    \item a \textbf{legislator performance layer}, which aggregates those bill-level impacts according to how legislators
          sponsor, co-sponsor, and vote on them.
\end{enumerate}

Although this paper focuses primarily on the institutional and data environment of the United States, PoliScore is being
developed as an MIT-licensed, open framework that is not intrinsically tied to any single country. In principle, any
nation built on the rule of law---with machine-readable legislative text and a traceable record of lawmaker interactions---
could adopt, extend, or branch the PoliScore codebase. Doing so requires not only open-source licensing, but also careful
abstraction of jurisdiction-specific components (such as chamber structure, bill taxonomies, and voting procedures) and
thorough documentation so that new maintainers can adapt the core policy quality model to fit different constitutional
arrangements and political institutions without altering its foundational principles.

At present, PoliScore is a solo project built in spare time and sustained by personal conviction rather than grants or
institutional funding. The system described in the pages that follow is therefore aspirational: it outlines what could
exist if the necessary technical, civic, and financial support materialized. Throughout, the emphasis is on clarity and
transparency, so that others can critique, extend, and---if they wish---help build the infrastructure this vision implies.

%%%%%%%%%%%%%%%%%%%%%%%%%%%%%%%%%%%%%%%%%%%%%%%%%%%
%%%%%%%%%%%%%%%   SECTION 1   %%%%%%%%%%%%%%%%%%%%%
%%%%%%%%%%%%%%%%%%%%%%%%%%%%%%%%%%%%%%%%%%%%%%%%%%%

\section{Foundational Principles of Policy Quality}
\label{sec:foundations}

\subsection{Human Needs as the Basis of Societal Outcomes}

At the core of any political system lies a simple and empirically grounded truth: human beings have universal, predictable
needs. These include, at minimum, the requirements for survival (food, water, shelter, health, safety) and the conditions
for flourishing (education, opportunity, autonomy, stability, and participation in society).

This principle is not ideological. It traces broadly through the history of philosophy and public policy, from classical
notions of basic welfare to modern frameworks such as:

\begin{itemize}
    \item Maslow's hierarchy of needs (Maslow, 1943),
    \item Sen's capabilities approach (Sen, 1999),
    \item Rawlsian justice as fairness (Rawls, 1971),
    \item Nussbaum's list of central human capabilities (Nussbaum, 2006),
    \item the UN Human Development Index (UNDP, 1990),
    \item the OECD Better Life Index (OECD, 2011).
\end{itemize}

The implication is straightforward: policies ultimately exist to alter societal conditions in ways that affect these human
needs. Therefore, any concept of policy quality must begin with the recognition of these universal determinants of human
well-being.

\subsection{The Purpose of Public Policy: Maximizing Societal Benefit}

If human needs are universal, then public policy can be defined as:

\begin{quote}
\textit{A structured intervention intended to alter social, economic, or political conditions to maximize societal
well-being, minimize harm, and ensure long-term stability.}
\end{quote}

This definition integrates insights from multiple traditions:

\begin{itemize}
    \item Utilitarianism --- maximizing aggregate well-being.
    \item Rawlsian justice --- ensuring fairness and protection for the least advantaged.
    \item Capabilities theory --- expanding real freedoms and opportunities.
    \item Institutional economics --- ensuring efficiency, stability, and low transaction costs (North, 1990).
    \item Governance theory --- designing institutions that are transparent, accountable, and resilient (Ostrom, 1990).
\end{itemize}

From these traditions emerges a balanced, non-ideological goal:

\begin{quote}
\textit{Good policy is that which improves human outcomes without creating disproportionate harm, fragility, inequality,
or institutional dysfunction.}
\end{quote}

This is the foundational goal on which the PoliScore evaluative framework is built.

\subsection{From First Principles to a Standard Rubric of Policy Quality}

If
\begin{enumerate}[label=(\alph*)]
    \item humans have predictable needs, and
    \item the purpose of policy is to maximize societal benefit while minimizing harm,
\end{enumerate}
then it follows that policy quality can be systematically evaluated according to how effectively a proposal moves society
toward those outcomes.

PoliScore formalizes this into seven universal dimensions derived directly from these foundational principles:

\begin{enumerate}
    \item \textbf{Problem Clarity \& Causal Validity}\\
    Does the policy accurately diagnose the underlying issue and target the relevant causal mechanisms?

    \item \textbf{Evidence Base \& Empirical Support}\\
    Is the proposed intervention supported by empirical research, historical precedent, or meaningful comparative data?

    \item \textbf{Implementation Feasibility}\\
    Can existing institutions realistically execute the policy given resource, logistical, administrative, and temporal
    constraints?

    \item \textbf{Economic Efficiency \& Fiscal Sustainability}\\
    Does the policy use resources responsibly, minimize waste, and avoid unsustainable long-term obligations?

    \item \textbf{Distributional Impact \& Fairness}\\
    How are benefits and burdens distributed across populations, and does the policy unjustifiably disadvantage certain
    groups?

    \item \textbf{Governance Integrity \& Institutional Risk}\\
    Does the policy maintain transparency, accountability, and resilience while minimizing opportunities for corruption or
    abuse?

    \item \textbf{Unintended Consequences \& Systemic Risk}\\
    Does the policy introduce fragility, perverse incentives, or cascading failures that undermine the intended outcomes?
\end{enumerate}

These seven pillars are not ideological criteria. They are derived from the intersection of:

\begin{itemize}
    \item moral philosophy,
    \item economics,
    \item development theory,
    \item organizational behavior,
    \item risk analysis,
    \item governance studies,
    \item institutional design.
\end{itemize}

Taken together, they form a non-partisan, foundational theory of policy quality designed explicitly for computational
and rubric-based evaluation.

\subsection{Why This Framework Is Necessary}

Existing institutions (such as the CBO, independent economic modeling groups, think tanks, and academic departments)
evaluate policy through isolated lenses:

\begin{itemize}
    \item cost impacts,
    \item economic forecasts,
    \item ideological alignment,
    \item advocacy positions,
    \item program evaluation after implementation.
\end{itemize}

None provide a unified, interdisciplinary, pre-implementation standard for determining whether legislation is well-designed,
feasible, fair, evidence-based, and structurally beneficial.

PoliScore seeks to fill this gap by offering:

\begin{itemize}
    \item a consistent methodology,
    \item grounded in decades of cross-disciplinary research,
    \item applicable directly to bill text,
    \item measurable, reproducible, and benchmarkable,
    \item independent of ideology.
\end{itemize}

This framework is the intellectual foundation for the PoliScore grading system and the PoliBench benchmark suite.

\subsection{A New Field: Policy Quality Engineering}

By synthesizing philosophical foundations, institutional economics, governance theory, and modern AI evaluation, PoliScore
introduces what is effectively a new discipline:

\begin{quote}
\textbf{Policy Quality Engineering:} the systematic, reproducible evaluation of public policy according to universal human
needs, societal benefit, and institutional feasibility.
\end{quote}

This document is ultimately an invitation to help build that field. The rest of the paper develops both the theoretical
and practical aspects of this discipline and outlines how it could be extended from bill-level analysis to legislator-level
performance grading, given sufficient technical and institutional support.

%%%%%%%%%%%%%%%%%%%%%%%%%%%%%%%%%%%%%%%%%%%%%%%%%%%
%%%%%%%%%%%%%%%   SECTION 2   %%%%%%%%%%%%%%%%%%%%%
%%%%%%%%%%%%%%%%%%%%%%%%%%%%%%%%%%%%%%%%%%%%%%%%%%%

\section{The PoliScore Framework}
\label{sec:framework}

The PoliScore framework provides a systematic, non-partisan method for evaluating the quality of public policy based on
its expected real-world impact, feasibility, and alignment with universal human needs. Derived from the foundational
principles articulated in Section~\ref{sec:foundations}, the framework operationalizes these concepts through seven core
evaluation dimensions, each representing a necessary component of high-quality legislation.

Each dimension is designed to be:
\begin{itemize}
    \item philosophically grounded,
    \item empirically motivated,
    \item institutionally relevant,
    \item computationally assessable, and
    \item applicable directly to legislative text.
\end{itemize}

Together, these dimensions form a comprehensive rubric for determining whether a policy is constructed in a way that
maximizes societal benefit while minimizing harm, inefficiency, and unintended consequences.

\subsection{Problem Clarity \& Causal Validity}

Effective policy begins with a clear, accurate understanding of the problem it seeks to address. Vague or misdiagnosed
problems lead to interventions that fail to produce meaningful improvements or that target symptoms rather than causes.

A policy demonstrates clarity and causal validity when:
\begin{itemize}
    \item the problem is explicitly defined and empirically measurable;
    \item the underlying causal mechanisms are identified;
    \item the proposed intervention plausibly affects those mechanisms;
    \item the theory of change is coherent and logically sound.
\end{itemize}

Policies that rest on untested assumptions, moral panic, or ideological narratives---rather than a valid causal model---
score poorly on this dimension.

\subsection{Evidence Base \& Empirical Support}

High-quality policy proposals demonstrate clear grounding in empirical research, comparative case studies, or validated
theoretical frameworks. This dimension evaluates whether the intervention is supported by evidence that similar approaches
have succeeded elsewhere, or whether its expected outcomes are consistent with existing knowledge.

Relevant criteria include:
\begin{itemize}
    \item citation of empirical findings or documented precedents;
    \item alignment with established best practices in relevant fields;
    \item avoidance of claims contradicted by available data;
    \item transparency about uncertainty and knowledge gaps.
\end{itemize}

Policies lacking empirical grounding may rely on wishful thinking or unproven assumptions, increasing the risk of unintended
harm.

\subsection{Implementation Feasibility}

Even a theoretically sound policy can fail if it is infeasible to implement. Feasibility depends on the capacity of
institutions, agencies, and local systems to carry out the policy's mandates with available resources, logistics, workforce,
technology, and time.

This dimension evaluates:
\begin{itemize}
    \item administrative complexity and burden;
    \item clarity of agency responsibilities;
    \item resource requirements (financial, human, technical);
    \item timeline realism;
    \item reliance on unavailable or overextended infrastructure;
    \item potential bottlenecks or bureaucratic overload.
\end{itemize}

Policies that impose unrealistic workloads, require nonexistent infrastructure, or centralize responsibility in ways that
exceed institutional capacity receive low scores.

\subsection{Economic Efficiency \& Fiscal Sustainability}

Public policy must allocate resources responsibly, avoid generating structural inefficiencies, and maintain long-term fiscal
viability. This dimension evaluates whether the benefits of the policy justify its costs, whether incentives are aligned
with real-world behaviors, and whether it avoids unnecessary waste or economic distortion.

Considerations include:
\begin{itemize}
    \item long-term funding stability;
    \item cost-benefit alignment;
    \item administrative overhead;
    \item market distortions or inefficiencies;
    \item externalities (positive or negative);
    \item dynamic economic effects and sustainability over time.
\end{itemize}

Policies that rely on implausible revenue assumptions, produce excessive deadweight loss, or generate persistent deficits
score poorly.

\subsection{Distributional Impact \& Fairness}

Public policies distribute benefits and burdens across different groups within society. A high-quality policy should not
impose disproportionate harm on specific populations or create unjustifiable imbalances in who gains and who loses. This
dimension evaluates how a policy's effects are spread across income levels, regions, industries, and demographic groups,
and whether these effects align with broadly accepted principles of fairness and responsible governance.

Key considerations include:
\begin{itemize}
    \item which groups receive the primary benefits of the policy;
    \item which groups bear the costs or risks;
    \item whether the distribution of impacts is reasonable and transparent;
    \item whether the policy inadvertently worsens existing disadvantages;
    \item whether the policy shifts burdens onto populations with limited capacity to absorb them.
\end{itemize}

This pillar does not require or assume equal outcomes. Instead, it assesses whether the distribution of impacts is
justified, defensible, and consistent with the stated goals of the policy, and whether any imbalances introduce meaningful
risk or undue harm.

\subsection{Governance Integrity \& Institutional Risk}

Policies exist within complex governance structures. This dimension evaluates whether a proposal strengthens or undermines
institutional integrity, transparency, accountability, and the rule of law.

Key criteria:
\begin{itemize}
    \item clarity of authority and decision-making processes;
    \item adequacy of oversight and accountability mechanisms;
    \item risks of corruption, abuse of power, or regulatory capture;
    \item concentration of unregulated authority;
    \item resilience to political manipulation;
    \item clarity in compliance requirements.
\end{itemize}

Policies that concentrate discretionary power in unaccountable agencies, lack oversight, or create opportunities for
corruption receive lower scores.

\subsection{Unintended Consequences \& Systemic Risk}

Complex systems often respond unpredictably to policy interventions. This dimension assesses the extent to which a policy
may produce harmful unintended consequences, including perverse incentives, moral hazard, market failures, bureaucratic
overload, or cascading systemic risks.

Evaluative criteria include:
\begin{itemize}
    \item creation of fragile dependencies;
    \item incentive misalignment;
    \item spillovers into adjacent systems;
    \item risk of black markets or evasion;
    \item increased systemic fragility or bottlenecks;
    \item insufficient fail-safes or fallback mechanisms.
\end{itemize}

Policies that appear beneficial in theory but introduce hidden structural costs or vulnerabilities receive lower scores.

\subsection{Sectoral Impact Model and Overall Impact to Society}
\label{subsec:sectoral}

The seven pillars described above characterize \emph{structural policy quality}. To connect these abstractions to concrete
societal outcomes, PoliScore introduces a sectoral impact model and a scalar ``Overall Impact to Society'' score for each
bill.

For each bill \(b\), PoliScore defines a set of sectoral impact scores:
\[
I_{b,s} \in [-100, 100] \cup \{\text{N/A}\}
\]
for each sector \(s\) in a fixed set \(\mathcal{S}\) that includes, at minimum:

\begin{itemize}
    \item Agriculture and Food
    \item Education
    \item Transportation
    \item Economics and Commerce
    \item Foreign Relations
    \item Government Efficiency and Management
    \item Healthcare
    \item Housing
    \item Energy
    \item Technology
    \item Immigration
    \item National Defense
    \item Crime and Law Enforcement
    \item Wildlife and Forest Management
    \item Public Lands and Natural Resources
    \item Environmental Management and Climate Change
\end{itemize}

Each sector score is rated from \(-100\) (very harmful) to \(0\) (neutral) to \(+100\) (very helpful), or marked as N/A
when the sector is not meaningfully affected by the bill.

From this sectoral vector, PoliScore defines an ``Overall Impact to Society'' score:
\[
I_b^{\ast} \in [-100, 100],
\]
representing a synthetic estimate of the bill's aggregate impact on societal well-being, considering the interaction of
all affected sectors.

In practice, \(I_{b,s}\) and \(I_b^{\ast}\) are produced by an evaluator (currently an AI model in prototype experiments)
that:

\begin{enumerate}
    \item reads the bill text,
    \item applies the seven-pillar framework to understand structure, feasibility, and risks,
    \item performs constrained research where appropriate, and
    \item assigns sectoral scores and an overall score with an accompanying natural-language justification.
\end{enumerate}

The sectoral model and overall score serve as the bridge between abstract policy quality and concrete societal impact,
and they are the primary inputs into legislator-level aggregation described in Section~\ref{sec:pipeline}.

\subsection*{Summary of the Framework}

Together, these seven structural dimensions and the sectoral / overall impact model form a holistic, first-principles
representation of policy quality. They address not only a policy's intent and potential benefits, but also its feasibility,
fairness, evidence base, institutional risks, and long-term sustainability.

The PoliScore framework is designed to:
\begin{itemize}
    \item provide structured, rational evaluation of legislation;
    \item enable reproducible scoring across policies and time;
    \item support AI-assisted and human-driven legislative analysis;
    \item inform policymakers, researchers, and the public;
    \item reduce reliance on ideological or partisan heuristics.
\end{itemize}

This framework also provides the theoretical foundation for both the bill-to-legislator pipeline (Section~\ref{sec:pipeline})
and the PoliBench benchmark suite (Section~\ref{sec:polibench}), which tests whether AI systems can reliably interpret and
evaluate policy quality along these dimensions.

\paragraph{Definition (Intended Societal Impact).}
\emph{Intended Societal Impact (ISI)} is the aggregate, expected effect on society implied by the set of bills a legislator
sponsors, co-sponsors, or votes for or against, weighted by interaction type. ISI represents the legislator’s revealed
policy intent under the PoliScore evaluation framework: the world their legislative record points toward, assuming their
positions were enacted.

%%%%%%%%%%%%%%%%%%%%%%%%%%%%%%%%%%%%%%%%%%%%%%%%%%%
%%%%%%%%%%%%%%%   PIPELINE   %%%%%%%%%
%%%%%%%%%%%%%%%%%%%%%%%%%%%%%%%%%%%%%%%%%%%%%%%%%%%

\section{From Bills to Legislators: The PoliScore Impact Pipeline}
\label{sec:pipeline}

PoliScore is ultimately designed to answer two questions:

\begin{enumerate}
    \item \textbf{Bill-level question:} What is the predicted impact of this bill on society?
    \item \textbf{Legislator-level question:} Over the course of their recent history, what has this legislator actually
    done for (or to) the public through their legislative actions?
\end{enumerate}

This section describes the multi-stage pipeline that connects the seven-pillar framework and sectoral impact model to
legislator performance scores and letter grades. As of today, parts of this pipeline exist in prototype form; the full
design is presented here as a target architecture for a more mature, well-resourced system.

\subsection{High-Level Pipeline Overview}

Conceptually, the end-to-end pipeline can be summarized as:

\begin{enumerate}
    \item Bill text is segmented and analyzed under the PoliScore framework.
    \item A sectoral impact vector and an ``Overall Impact to Society'' score are produced for each bill.
    \item Legislator--bill interactions (sponsorship, co-sponsorship, and votes) are collected.
    \item Each legislator's Intended Societal Impact score is computed as a weighted function of the overall bill impact
          scores.
    \item Legislators and bills receive letter grades derived from their scalar impact scores.
    \item Parameterized natural-language summaries translate these metrics into human-readable narratives.
\end{enumerate}

Figure~\ref{fig:pipeline} depicts this process.

\begin{figure}[h]
\centering
\begin{tikzpicture}[node distance=4cm, >=stealth,
    every node/.style={rectangle, rounded corners, draw, align=center, width=3cm, minimum height=1cm}]
    \node (bills) {Legislative\\Text};
    \node (pillars) [right of=bills] {Seven-Pillar\\Policy Quality Analysis};
    \node (sectors) [right of=pillars] {Sectoral Impact\\Scores};
    \node (overall) [right of=sectors] {Overall Impact\\to Society};
    \node (leg) [right of=overall] {Legislator\\Intended Societal\\Impact \& Grades};

    \draw[->] (bills) -- (pillars);
    \draw[->] (pillars) -- (sectors);
    \draw[->] (sectors) -- (overall);
    \draw[->] (overall) -- (leg);
\end{tikzpicture}
\caption{High-level PoliScore pipeline from bill text to legislator grades.}
\label{fig:pipeline}
\end{figure}

\noindent
Throughout this paper, a legislator's score is framed primarily in terms of
\textbf{Intended Societal Impact}. This quantity represents the aggregate,
expected effect on society implied by the legislator's pattern of sponsorships,
co-sponsorships, and votes, assuming their preferred positions on bills were
consistently enacted. It therefore reflects the legislator's \emph{revealed
policy intent} and the structural quality of the policies they support, rather
than the empirical outcomes of laws that were actually implemented. In
Section~\ref{subsec:alt-metrics} we introduce complementary metrics---such as
realized impact and pillar- or sector-level performance---that provide additional
perspectives on legislative behavior.

While this pipeline is currently prototyped using AI systems as the primary evaluators, each step is defined in a way that
could be executed by trained human analysts following the same rubric, or scaled up by future AI systems that have been
explicitly validated on PoliBench.

\subsection{Bill-Level Scoring Prompt and Outputs}

In its early instantiation, PoliScore uses a structured evaluation template for bill-level scoring. The evaluator (AI model
or human) is asked to behave as a non-partisan oversight committee, reading full bill text and producing:

\begin{itemize}
    \item sectoral impact scores \(I_{b,s}\),
    \item an ``Overall Impact to Society'' score \(I_b^{\ast}\),
    \item short and long-form explanatory reports, and
    \item a self-rated confidence score.
\end{itemize}

The template for the Stats section is:

\begin{quote}
\texttt{Stats:}\\
\texttt{Score the following bill on the estimated impact to the United States upon the following criteria, rated from -100 (very harmful) to 0 (neutral) to +100 (very helpful) or N/A if it is not relevant.}\\

\texttt{Agriculture and Food: <score or N/A>}\\
\texttt{Education: <score or N/A>}\\
\texttt{Transportation: <score or N/A>}\\
\texttt{Economics and Commerce: <score or N/A>}\\
\texttt{Foreign relations: <score or N/A>}\\
\texttt{Government Efficiency and Management: <score or N/A>}\\
\texttt{Healthcare: <score or N/A>}\\
\texttt{Housing: <score or N/A>}\\
\texttt{Energy: <score or N/A>}\\
\texttt{Technology: <score or N/A>}\\
\texttt{Immigration: <score or N/A>}\\
\texttt{National Defense: <score or N/A>}\\
\texttt{Crime and Law Enforcement: <score or N/A>}\\
\texttt{Wildlife and Forest Management: <score or N/A>}\\
\texttt{Public Lands and Natural Resources: <score or N/A>}\\
\texttt{Environmental Management and climate change: <score or N/A>}\\
\texttt{Overall Impact to society: <score or N/A>}
\end{quote}

Additional sections provide a descriptive title, a short report, a long report referencing specific bill provisions, and a
numeric confidence score. In a fully realized implementation, these outputs would be logged, versioned, and made publicly
inspectable.

\subsection{Aggregating Sectoral Scores into Overall Impact}

The ``Overall Impact to Society'' score \(I_b^{\ast}\) may be produced in one of two ways:

\begin{enumerate}
    \item \textbf{Direct overall assessment.} The evaluator directly assigns \(I_b^{\ast}\) based on holistic reasoning over the bill and its context.
    \item \textbf{Derived aggregation.} The evaluator provides only sectoral scores \(I_{b,s}\), and the system computes:
    \[
    I_b^{\ast} = f(\{I_{b,s}\}_{s \in \mathcal{S}}),
    \]
    where \(f\) is an aggregation function (e.g., a weighted average of non-N/A sector scores).
\end{enumerate}

In practice, PoliScore can use both approaches simultaneously: the evaluator provides a direct overall score, and the
system cross-checks it against a derived aggregate. Discrepancies can be logged for analysis, calibration, or human review.

A simple, transparent aggregation rule is:

\[
I_b^{\ast} = \frac{\sum_{s \in \mathcal{S}_{b}} w_s I_{b,s}}{\sum_{s \in \mathcal{S}_{b}} |w_s|},
\]
where \(\mathcal{S}_{b}\) is the set of sectors marked as relevant (non-N/A) for bill \(b\), and \(w_s\) are sector weights.
In the earliest implementation, \(w_s = 1\) for all sectors, yielding an unweighted average. More sophisticated variants
may:

\begin{itemize}
    \item estimate \(w_s\) from public opinion and expert surveys;
    \item tune \(w_s\) based on empirical outcomes of historical legislation;
    \item incorporate robustness constraints (e.g., penalize extreme scores confined to a single sector).
\end{itemize}

The key requirement is that the mapping from sectoral scores to overall impact is formally defined, transparent, and stable
over the evaluation period.

\subsection{Legislator--Bill Interaction Model}

Legislators interact with bills in recognizable ways: they sponsor them, co-sponsor them, and vote for or against them.
Let:

\begin{itemize}
    \item \(\mathcal{B}\) denote the set of bills,
    \item \(\mathcal{L}\) denote the set of legislators,
    \item \(I_b^{\ast}\) denote the overall impact score of bill \(b\),
    \item \(T \in \{\text{Sponsor}, \text{CoSponsor}, \text{VotedFor}, \text{VotedAgainst}\}\) denote the type of interaction.
\end{itemize}

PoliScore defines weights \(w_T\) for each interaction type, reflecting both normative judgment and intuitive voter
expectations. A simple weighting scheme currently used in practice is:

\begin{align*}
w_{\text{Sponsor}} &= 1.0, \\
w_{\text{CoSponsor}} &= 0.7, \\
w_{\text{VotedFor}} &= 0.5, \\
w_{\text{VotedAgainst}} &= -0.5.
\end{align*}

The negative weight for \(\text{VotedAgainst}\) indicates that opposing a harmful bill (\(I_b^{\ast} < 0\)) generates
positive credit, while opposing a beneficial bill (\(I_b^{\ast} > 0\)) generates negative credit.

For a given legislator \(\ell \in \mathcal{L}\), let \(\mathcal{I}_{\ell}\) denote the set of all (bill, interaction-type)
pairs in which they participate. The legislator's Intended Societal Impact score, \(\text{ISI}_{\ell}\), is defined as:

\[
\text{ISI}_{\ell} =
\frac{\sum_{(b, T) \in \mathcal{I}_{\ell}} w_T \, I_b^{\ast}}
     {\sum_{(b, T) \in \mathcal{I}_{\ell}} |w_T|}.
\]

This formulation:

\begin{itemize}
    \item credits legislators for sponsoring and supporting beneficial bills,
    \item debits them for supporting harmful bills,
    \item rewards them for opposing harmful bills,
    \item penalizes them for opposing beneficial bills,
    \item normalizes scores so that legislators with many interactions are comparable to those with fewer.
\end{itemize}

Intended Societal Impact should be interpreted as the \emph{expected societal impact of the legislator's policy record if
their preferred positions were enacted}. It reflects intent, judgment, and policy quality alignment under the PoliScore
framework.

\subsection{Alternative and Complementary Legislator Metrics}
\label{subsec:alt-metrics}

The aggregation rule above defines a legislator’s primary score as a function of bill-level impact values. In PoliScore,
this quantity is called the legislator’s \textbf{Intended Societal Impact} (ISI). While ISI is the central metric, the
same pipeline naturally supports additional views of legislative performance.

This subsection introduces three complementary perspectives built on the same primitives: realized societal impact,
pillar-level performance, and sector-level performance.

\subsubsection{Realized Societal Impact}

While ISI measures the consequences of a legislator’s \emph{intent}, a natural complement is to measure their contribution
to laws that actually entered into force. Let $E_b \in \{0,1\}$ indicate whether bill $b$ was enacted. A realized variant
is then:

\[
\text{RealizedImpact}_{\ell} =
\frac{\sum_{(b, T) \in \mathcal{I}_{\ell}} E_b \, w_T \, I_b^{\ast}}
     {\sum_{(b, T) \in \mathcal{I}_{\ell}} E_b \, |w_T|}.
\]

Where ISI captures ``the world they are trying to create,'' the realized score captures ``the world their successful
actions have contributed to.'' The two together provide a fuller behavioral portrait.

\subsubsection{Pillar-Level Performance Profiles}

Each bill $b$ is evaluated along seven structural pillars of policy quality.
Let $D_{b,k}$ denote the score of bill $b$ on pillar $k \in \{1,\dots,7\}$. A
legislator-level pillar profile is defined by:

\[
Q_{\ell,k} =
\frac{\sum_{(b, T) \in \mathcal{I}_{\ell}} w_T \, D_{b,k}}
     {\sum_{(b, T) \in \mathcal{I}_{\ell}} |w_T|}.
\]

The vector $(Q_{\ell,1}, \ldots, Q_{\ell,7})$ summarizes how closely a
legislator’s supported policies align with high-quality design principles:
clarity of problem definition, evidence base, implementation feasibility,
fairness, governance integrity, and systemic risk. These profiles can be
surfaced directly to users (e.g., as tables or radar plots) and incorporated
into summary narratives.

\subsubsection{Sectoral Performance Profiles}

Analogously, for each bill PoliScore defines sectoral impacts $I_{b,s}$ for
sectors $s \in \mathcal{S}$. Aggregating these yields:

\[
R_{\ell,s} =
\frac{\sum_{(b, T) \in \mathcal{I}_{\ell}} w_T \, I_{b,s}}
     {\sum_{(b, T) \in \mathcal{I}_{\ell}} |w_T|}.
\]

The vector $(R_{\ell,s})_{s \in \mathcal{S}}$ reveals where a legislator is most
helpful or harmful across domains such as healthcare, energy, national defense,
housing, or the environment.

Together, the Intended Societal Impact score, its realized-impact counterpart,
and these pillar- and sector-level profiles form a multi-dimensional,
transparent representation of legislative performance grounded in the
PoliScore policy quality framework.

\subsection{Letter Grades for Bills and Legislators}

To make the results accessible to voters, PoliScore converts raw impact scores into letter grades.

For bills, the letter grade is derived directly from \(I_b^{\ast}\); for legislators, from \(\text{ISI}_{\ell}\) or from
an alternative legislator-level scalar such as \(\text{RealizedImpact}_{\ell}\). A simple and currently deployed mapping
is:

\begin{align*}
    \text{A} &: I \geq 40, \\
    \text{B} &: 30 \leq I < 40, \\
    \text{C} &: 15 \leq I < 30, \\
    \text{D} &: 0 \leq I < 15, \\
    \text{F} &: I < 0,
\end{align*}
where \(I\) is either a bill's overall impact score \(I_b^{\ast}\) or a legislator's aggregate score.

These thresholds are intentionally simple and interpretable. They are not meant to be perfect statistical constructs, but
to provide a consistent, monotonic mapping between scalar impact scores and an intuitive grading scale familiar to the
public.

\subsection{Parameterized Legislator Summaries}

Numerical scores and letter grades provide clarity and comparability, but they do not by themselves tell a story. To bridge
this gap, PoliScore uses parameterized natural-language prompts that consume:

\begin{itemize}
    \item a legislator's aggregate impact statistics by sector,
    \item their overall letter grade,
    \item a curated list of their most consequential bill interactions.
\end{itemize}

A generalized template is:

\begin{quote}
\texttt{The United States \{\{politicianType\}\} \{\{fullName\}\} has been evaluated based on recent legislative performance and has received the following policy area grades (scores range from -100 to 100):}\\

\texttt{\{\{stats\}\}}\\[0.5em]
\texttt{Based on these scores, this legislator has received the overall letter grade: \{\{letterGrade\}\}. You will be given bill interaction summaries of this politician's recent legislative history, sorted by their impact to the relevant policy area grades. Please generate a layman's, concise, three paragraph, \{\{analysisType\}\}, highlighting any \{\{behavior\}\}, identifying trends, referencing specific bill titles (in quotes), and pointing out major focuses and priorities of the legislator. Focus on the policy areas with the largest score magnitudes (either positive or negative). Do not include the legislator's policy area grade scores and do not mention their letter grade in your summary.}\\

\texttt{\{\{billInteractions\}\}}
\end{quote}

Where:

\begin{itemize}
    \item \texttt{politicianType} is ``Senator'' or ``House Representative'',
    \item \texttt{fullName} is the legislator's name,
    \item \texttt{stats} is a textual rendering of sectoral scores,
    \item \texttt{letterGrade} is the overall grade,
    \item \texttt{analysisType} is one of ``endorsement'', ``mixed analysis'', or ``harsh critique'', chosen based on the letter grade,
    \item \texttt{behavior} describes whether to emphasize accomplishments, alarming behavior, or both,
    \item \texttt{billInteractions} is a list of the legislator's most impactful bill interactions by sector and magnitude.
\end{itemize}

This design allows the system to vary tone in a controlled, transparent way while keeping the underlying scoring rules
fixed. In a fully realized implementation, these summaries would be clearly labeled as model-generated analysis grounded
in the underlying metrics, not as neutral journalism.

\subsection{AI Behavior, Training Data, and Non-Partisan Prompts}

PoliScore's bill-level evaluations are intended to be non-partisan and rooted in ``Overall Impact to Society''. Three
properties of modern language models make this goal plausible when combined with careful prompt design:

\begin{enumerate}
    \item \textbf{Frequency weighting.} Model predictions are influenced by patterns frequently observed in training data.
    Because training corpora contain large volumes of books, encyclopedias, and mainstream reporting, many of the model's
    ``default'' positions track the most commonly articulated views in those sources.

    \item \textbf{Contextual weighting.} Given a specific prompt, the model preferentially draws from portions of its
    training data that match the requested context (e.g., technical analysis vs.\ jokes). When prompted as a non-partisan
    oversight committee asked to cite scientific studies, official reports, and expert opinions, the model tends to
    prioritize more authoritative and technical contexts.

    \item \textbf{Coherence pressure.} To answer PoliScore-style prompts, a model must synthesize scattered facts and
    arguments into a single coherent narrative that satisfies multiple constraints (sectoral scoring, justification,
    trade-off analysis). This encourages structured reasoning rather than surface-level slogan repetition.
\end{enumerate}

PoliScore reinforces non-partisanship through explicit design choices:

\begin{itemize}
    \item prompts emphasize ``Overall Impact to Society'' rather than partisan advantage,
    \item the aggregation logic is mechanically neutral and publicly documented,
    \item the code is released under a permissive open-source license, inviting scrutiny and forks,
    \item funding and governance of the project aim to remain independent of party structures,
    \item outputs are transparent and inspectable both at bill and legislator levels.
\end{itemize}

At the same time, AI systems can inherit biases from their training data, including policy preferences that align with
majority public or scientific opinion on issues such as renewable energy, reproductive rights, or gun policy. Rather than
denying this, PoliScore presents its outputs as a structured reflection of how a capable model, prompted to be non-partisan,
reconciles expert and public consensus with the specifics of legislative text. Users are encouraged to examine the
underlying justifications and decide for themselves whether they agree with the resulting grades.

%%%%%%%%%%%%%%%%%%%%%%%%%%%%%%%%%%%%%%%%%%%%%%%%%%%
%%%%%%%%%%%%%%% POLIBENCH %%%%
%%%%%%%%%%%%%%%%%%%%%%%%%%%%%%%%%%%%%%%%%%%%%%%%%%%

\section{The PoliBench Benchmark Suite}
\labelsec{polibench}\label{sec:polibench}

The PoliBench Benchmark Suite is a standardized set of tests designed to evaluate whether AI systems can accurately assess
public policy along the seven dimensions of the PoliScore Framework and reason coherently about sectoral and overall
societal impacts. Whereas PoliScore provides the conceptual model and pipeline for policy quality and legislator
performance, PoliBench operationalizes those ideas into concrete, reproducible tasks that measure an AI system's ability
to interpret legislative intent, feasibility, consequences, and institutional design.

PoliBench is not intended as a performance leaderboard for general AI capabilities. Rather, it is a domain-specific
benchmark focused on policy reasoning, causal inference, and institutional awareness---areas where existing language
models often demonstrate gaps despite strong natural language proficiency. The benchmark enables systematic comparison
across AI systems and provides an empirical foundation for evaluating progress in computational policy analysis.

\subsection{Motivation}

Despite rapid advances in large language models (LLMs), there is currently no standardized method for testing their
ability to interpret legislation or assess public policy quality. Existing AI benchmarks measure skills such as:

\begin{itemize}
    \item question answering (SQuAD, Natural Questions),
    \item general knowledge (MMLU),
    \item reasoning (GSM8K, ARC),
    \item truthfulness (TruthfulQA),
    \item code generation (HumanEval).
\end{itemize}

None of these capture the skills required for policy evaluation, such as:

\begin{itemize}
    \item recognizing ambiguous or misleading problem statements;
    \item detecting infeasible mandates;
    \item identifying governance risks;
    \item reasoning about distributional impacts and sectoral tradeoffs;
    \item understanding institutional constraints;
    \item anticipating unintended consequences;
    \item evaluating evidence claims in context.
\end{itemize}

Public policy is a systems problem involving economics, governance, logistics, human behavior, and institutional dynamics.
PoliBench fills a critical gap by testing whether AI systems can navigate these complexities in a disciplined and
consistent way.

\subsection{Objectives}

PoliBench is designed to achieve five key objectives:

\begin{enumerate}[label=(\arabic*)]
    \item \textbf{Evaluate policy-specific reasoning}\\
    Measure whether an AI system can analyze legislative text in ways aligned with the PoliScore pillars and sectoral
    impact model.

    \item \textbf{Provide reproducible, standardized tests}\\
    Ensure that models are evaluated under identical conditions, enabling meaningful comparisons.

    \item \textbf{Identify structural weaknesses in AI policy analysis}\\
    Pinpoint which dimensions (e.g., feasibility, unintended consequences) pose the greatest difficulty for current
    models.

    \item \textbf{Support model improvement}\\
    Provide researchers with targeted diagnostics for training and fine-tuning AI systems on legislative reasoning tasks.

    \item \textbf{Promote transparency and accountability}\\
    Allow policymakers, academics, and the public to understand the strengths and limitations of AI in this domain.
\end{enumerate}

\subsection{Benchmark Structure}

PoliBench is organized into seven test suites, one for each structural dimension of the PoliScore Framework, plus optional
suites for sectoral impact prediction and legislator-level aggregation consistency. Each suite contains multiple task
types, designed to assess both conceptual understanding and applied reasoning.

Core task families include:

\begin{itemize}
    \item problem clarity and causal validity,
    \item evidence base and empirical support,
    \item implementation feasibility,
    \item economic efficiency and fiscal sustainability,
    \item distributional impact and fairness,
    \item governance integrity and institutional risk,
    \item unintended consequences and systemic risk,
    \item sectoral and overall impact estimation (optional),
    \item bill-to-legislator aggregation reasoning (optional).
\end{itemize}

Each suite includes constructed counterexamples, fictional bills, paired policy comparisons, and scenario-based prompts
that allow objective scoring.

\subsection{Benchmark Format and Example Tasks}

Each PoliBench task is designed to be:

\begin{itemize}
    \item \emph{deterministic} --- clear pass/fail or graded criteria,
    \item \emph{grounded} --- tied to a specific quality dimension,
    \item \emph{text-based} --- directly applicable to legislative language,
    \item \emph{model-agnostic} --- usable by any AI system,
    \item \emph{transparent} --- accompanied by human-annotated rationale and expected answer patterns.
\end{itemize}

Tasks follow formats such as:

\begin{itemize}
    \item multiple-choice reasoning tests,
    \item short-form explanation tasks,
    \item bill snippet analysis,
    \item policy comparison tasks,
    \item error detection tasks (spot-the-flaw),
    \item numeric prediction of sectoral impacts for simplified scenarios.
\end{itemize}

Illustrative examples include:

\paragraph{Feasibility Example.} \mbox{}\\
\textbf{Prompt:} A bill requires every rural county (population $< 5{,}000$) to operate a full-service emergency hospital
within one year.\\
\textbf{Question:} Identify the primary implementation challenge.\\
\textbf{Expected:} Workforce shortages; infrastructure constraints; unrealistic timeline.

\paragraph{Governance Risk Example.} \mbox{}\\
\textbf{Prompt:} A bill grants an agency director unilateral authority to allocate funds ``as they see fit,'' without
reporting requirements.\\
\textbf{Expected:} Lack of oversight; corruption risk; unclear accountability.

\paragraph{Distributional Impact Example.} \mbox{}\\
\textbf{Prompt:} A tax credit is available only to households with mortgage interest payments.\\
\textbf{Expected:} Benefits flow primarily to homeowners; renters receive no support; regressive impact.

\subsection{Scoring and Evaluation}

Each model evaluated on PoliBench would receive:

\begin{itemize}
    \item per-suite scores (0--100),
    \item a composite PoliBench score,
    \item diagnostic reports identifying which dimensions require improvement.
\end{itemize}

Scores reflect:

\begin{itemize}
    \item correctness,
    \item reasoning quality,
    \item internal consistency,
    \item robustness across variations,
    \item sensitivity to subtle policy flaws.
\end{itemize}

By aggregating results across suites, PoliBench would reveal the policy reasoning profile of an AI model and help determine
whether it is suitable for deployment within the PoliScore evaluation pipeline.

%%%%%%%%%%%%%%%%%%%%%%%%%%%%%%%%%%%%%%%%%%%%%%%%%%%
%%%%%%%%%%%%%%%   METHODOLOGY   %%%%%%%%
%%%%%%%%%%%%%%%%%%%%%%%%%%%%%%%%%%%%%%%%%%%%%%%%%%%

\section{Methodology}
\label{sec:methodology}

The methodology underlying PoliScore is designed to translate legislative text into a structured evaluation across the
seven dimensions of policy quality, sectoral and overall societal impact, and finally legislator performance. The process
combines computational analysis, domain-specific prompt engineering, and rubric-based scoring to produce a transparent,
interpretable assessment of both policies and legislators.

Crucially, the methodology is \emph{process-first}: AI models are current instantiations of the evaluator, but the pipeline
itself is defined in a model-agnostic way that could be executed by trained human raters or alternative AI systems. At
present, many of these steps exist as prototypes or design documents rather than production-grade components; they are
described here to make the blueprint concrete and critiqueable.

The methodological approach consists of five integrated components:

\begin{enumerate}
    \item Text Preparation,
    \item Dimension-Level Policy Evaluation,
    \item Sectoral and Overall Impact Scoring,
    \item Legislator Aggregation and Grading,
    \item Interpretability and Justification.
\end{enumerate}

\subsection{Text Preparation}

Before analysis, the legislative proposal undergoes standardized preprocessing to ensure that the evaluator receives input
in a structured and interpretable form.

\subsubsection{Document Segmentation}

Legislation is segmented into logical units, such as:

\begin{itemize}
    \item sections and subsections,
    \item enumerated provisions,
    \item definitions,
    \item mandates,
    \item authorizations,
    \item appropriations.
\end{itemize}

Segmentation allows for focused analysis of complex bills and supports cross-referencing within explanations.

\subsubsection{Contextual Metadata}

Where available, contextual information is included:

\begin{itemize}
    \item policy domain (healthcare, tax policy, infrastructure, etc.),
    \item jurisdiction and legislative session,
    \item sponsoring entity,
    \item historical or comparative precedents,
    \item relevant statutory references.
\end{itemize}

Metadata helps evaluators identify feasibility constraints, institutional conflicts, and relevant comparators.

\subsubsection{Legislative Normalization}

Formatting inconsistencies, redundant boilerplate, and non-substantive artifacts (e.g., page headers, XML tags) are
removed to reduce noise. The goal is to present the evaluator with a clean representation of the substantive legal
content.

\subsection{Versioned Legislative Text and Semantic Diffing}

Real-world legislation is rarely static. Bills are frequently amended in committee, modified on the floor, reconciled
between chambers, or replaced entirely via ``strike-and-insert'' procedures. Treating a bill as a single, fixed document
therefore obscures an important part of the lawmaking process: \emph{how the text evolved over time and who supported
which changes}.

PoliScore can be extended to operate over \emph{versioned} legislative text rather than only the final enrolled version.
Given a sequence of bill versions \(\{b^{(1)}, b^{(2)}, \dots, b^{(T)}\}\), the system can:

\begin{itemize}
    \item compute traditional text diffs to identify added, removed, or modified provisions;
    \item apply \emph{semantic} diffing, using an evaluator to summarize the substantive effect of those changes (e.g.,
    ``narrows eligibility to these populations,'' ``removes the sunset clause,'' ``adds an enforcement mechanism'');
    \item update pillar scores and sectoral impacts incrementally as the bill evolves, rather than re-scoring from
    scratch;
    \item attribute amendments or version shifts to specific sponsors or coalitions, where metadata is available.
\end{itemize}

This opens the door to a richer class of analyses. Instead of asking only, ``What is the Overall Impact to Society of the
final bill text?'' PoliScore can also ask, ``How did the bill's Intended Societal Impact change over time, and which
actors consistently pushed it toward more or less beneficial outcomes?''

In principle, the same framework used for bill-level scoring can be applied to pairs of versions \((b^{(t)}, b^{(t+1)})\),
with the evaluator focused specifically on the marginal change in structure, feasibility, fairness, or systemic risk.
This enables semantic diffing that is far more informative than line-based text comparisons, and it aligns naturally with
the legislator-level aggregation described in Section~\ref{sec:pipeline}: amendments and versioned edits become additional
interaction events that shape a legislator's profile of Intended Societal Impact.

\subsection{Dimension-Level Policy Evaluation}

For each of the seven pillars of the PoliScore Framework, a structured evaluation prompt is applied. Each prompt is
designed to assess a specific dimension using criteria derived directly from the theoretical foundations in
Sections~\ref{sec:foundations} and \ref{sec:framework}.

Each dimension evaluation includes:

\begin{itemize}
    \item \textbf{targeted questions} probing specific failure modes,
    \item \textbf{rubric-aligned checklists} (e.g., for feasibility or governance risks),
    \item \textbf{requests for explicit trade-off analysis} where relevant.
\end{itemize}

The evaluator produces:

\begin{itemize}
    \item a short narrative assessment,
    \item a 0--100 numeric score for the dimension,
    \item optional flags indicating unusual uncertainty or potential contradictions with other dimensions.
\end{itemize}

Cross-pillar consistency checks (e.g., between feasibility, economic sustainability, and systemic risk) are applied to
detect internal conflicts.

\subsection{Sectoral and Overall Impact Scoring}

Dimension-level evaluations are conceptually upstream of sectoral scoring: understanding problem clarity, evidence base,
and unintended consequences informs a more realistic estimate of sector-level impacts.

The sectoral scoring step uses the Stats template and sectoral set \(\mathcal{S}\) described in
Section~\ref{subsec:sectoral}. The evaluator:

\begin{enumerate}
    \item identifies which sectors are meaningfully affected by the bill,
    \item assigns scores \(I_{b,s} \in [-100, 100]\) or N/A for each sector,
    \item produces an initial overall impact score \(I_b^{\ast}\),
    \item explains major positive and negative contributions with references to specific provisions.
\end{enumerate}

An aggregation function \(f\) is then applied as a cross-check or as a primary method if direct overall scoring is not
used. Discrepancies between direct and derived overall scores may trigger:

\begin{itemize}
    \item automatic confidence reduction,
    \item a request for reevaluation,
    \item human review for high-stakes bills.
\end{itemize}

\subsection{Legislator Aggregation and Grading}

Once bill-level overall impact scores \(I_b^{\ast}\) are computed, legislator performance can be evaluated using the
interaction model defined in Section~\ref{sec:pipeline}. The steps are:

\begin{enumerate}
    \item Construct the set of legislator--bill interactions from roll-call data and bill metadata (sponsorship and
    co-sponsorship).
    \item Apply the weighting scheme \(w_T\) to each interaction type \(T\).
    \item For each legislator \(\ell\), compute the Intended Societal Impact score \(\text{ISI}_{\ell}\) as a normalized
    weighted sum of \(I_b^{\ast}\).
    \item Optionally compute complementary metrics such as \(\text{RealizedImpact}_{\ell}\), pillar-level profiles
    \(Q_{\ell,k}\), and sectoral profiles \(R_{\ell,s}\).
    \item Map scalar legislator scores to letter grades via the standard threshold function.
    \item Generate sectoral breakdowns of the legislator's performance (e.g., average impact in healthcare vs.\ energy).
    \item Feed these statistics into the parameterized summary prompt to produce a narrative overview.
\end{enumerate}

Because all steps are explicit and documented, users can:

\begin{itemize}
    \item inspect which bills contributed most to a legislator's score,
    \item understand how interaction types were weighted,
    \item recompute scores under alternative weighting schemes if desired.
\end{itemize}

\subsection{Open-Source Implementation and International Adaptation}

Although the conceptual framework of PoliScore is jurisdiction-agnostic, any concrete implementation must contend with the
specific institutional details of a given legal system: chamber structures, bill numbering schemes, committee processes,
roll-call formats, and sector taxonomies. To make adaptation feasible across countries, PoliScore is being developed and
released under the permissive MIT license. This choice is deliberate: it allows governments, civil society organizations,
researchers, and independent developers to freely reuse, modify, and redistribute the codebase, including in proprietary
or jurisdiction-specific deployments, while preserving attribution.

In practical terms, an internationally extensible PoliScore implementation must:

\begin{itemize}
    \item separate core evaluation logic (the seven pillars, sectoral aggregation, legislator interaction model, and scoring
    rules) from jurisdiction-specific adapters for:
    \begin{itemize}
        \item ingesting legislative text and metadata,
        \item mapping local committee and chamber structures,
        \item interpreting session calendars and bill lifecycles,
        \item normalizing roll-call data and sponsor/co-sponsor records;
    \end{itemize}
    \item expose clear interfaces for customizing:
    \begin{itemize}
        \item sector lists and weights,
        \item default interaction weights for sponsorship and votes,
        \item naming conventions and role labels for legislators;
    \end{itemize}
    \item provide comprehensive documentation and worked examples showing how to stand up a new instance for a different
    parliament, congress, or assembly.
\end{itemize}

Because the MIT license imposes minimal restrictions, institutions in other rule-of-law democracies can choose to either
extend the mainline project (contributing back improvements and new adapters) or maintain their own branch tailored to
local norms and legal structures. In both cases, the underlying policy quality definitions and aggregation mathematics
can remain shared, enabling cross-national comparison of bills and legislative behavior while respecting jurisdictional
differences in constitutional design and political practice.

\subsection{Interpretability and Justification}

PoliScore emphasizes transparency and interpretability to ensure users understand \emph{why} a policy or legislator
received a given score.

\subsubsection{Bill-Level Explanations}

For each bill, PoliScore aims to produce:

\begin{itemize}
    \item a short report summarizing goals and expected impacts,
    \item a longer, lay-accessible report referencing concrete sections,
    \item a per-dimension breakdown of strengths and weaknesses,
    \item a justification for each sectoral and overall impact score.
\end{itemize}

\subsubsection{Legislator-Level Explanations}

For each legislator, PoliScore would provide:

\begin{itemize}
    \item the Intended Societal Impact score and letter grade,
    \item sectoral performance statistics,
    \item a list of most influential positive and negative bill interactions,
    \item a three-paragraph narrative summary generated from the parameterized prompt.
\end{itemize}

These outputs are intended to empower voters to ask informed questions, not to dictate specific political conclusions.

\subsubsection{Consistency and Robustness Checks}

To improve trustworthiness, PoliScore incorporates:

\begin{itemize}
    \item redundant prompts phrased in different ways to test stability,
    \item spot checks of bill and legislator scores under minor variations,
    \item adversarial tasks drawn from PoliBench,
    \item manual review and correction for anomalous outputs.
\end{itemize}

\subsection*{Summary}

The PoliScore methodology provides a rigorous, structured, and interpretable approach to evaluating both policy quality
and legislative performance. By combining legislative segmentation, dimension-specific evaluation, sectoral and overall
impact scoring, and explicit aggregation rules, PoliScore offers a transparent and reproducible system for assessing the
strengths and weaknesses of both bills and the legislators who interact with them. Realizing this methodology at scale
will require sustained engineering work, data infrastructure, and institutional stewardship that go far beyond the current
unfunded prototype.

%%%%%%%%%%%%%%%%%%%%%%%%%%%%%%%%%%%%%%%%%%%%%%%%%%%
%%%%%%%%%%%%%%%   COMPARISON   %%%%%%%%%
%%%%%%%%%%%%%%%%%%%%%%%%%%%%%%%%%%%%%%%%%%%%%%%%%%%

\section{Comparison to Existing Institutions}
\label{sec:comparison}

Public policy evaluation is not a new endeavor. Governments, academic centers, think tanks, and international organizations
have long attempted to analyze the consequences of legislation. However, their approaches are fragmented, domain-specific,
and often limited to economic forecasting, post-hoc program evaluation, or ideologically motivated analysis.

This section compares PoliScore to institutions most commonly involved in policy assessment, clarifying how the envisioned
framework would complement existing tools while filling an unmet need in pre-implementation policy quality evaluation and
systematic legislator performance grading.

\subsection{Congressional Budget Office (CBO)}

The Congressional Budget Office provides cost estimates and economic projections for federal legislation. Its analyses are
widely respected and intentionally nonpartisan. However, by statutory mandate, the CBO:

\begin{itemize}
    \item does not evaluate whether a policy is fair, feasible, or well-designed;
    \item does not assess governance risks or institutional fragility;
    \item does not analyze unintended consequences outside fiscal or macroeconomic domains;
    \item does not provide judgments about whether a policy is ``good'' or ``poor'';
    \item focuses almost exclusively on budgetary impacts, not policy quality or legislator performance.
\end{itemize}

\paragraph{How PoliScore differs.}

In its envisioned form, PoliScore would:

\begin{itemize}
    \item evaluate seven dimensions of policy quality rather than a single economic dimension;
    \item focus on pre-implementation design soundness and sectoral societal impacts;
    \item assess governance structure, feasibility, fairness, and systemic risk;
    \item aggregate bill impacts into legislator performance scores using transparent rules.
\end{itemize}

CBO answers: \textit{``What will this cost?''}\\
PoliScore aims to answer: \textit{``Is this policy structurally sound and societally beneficial, and how does a legislator's
record aggregate across such policies?''}

\subsection{Think Tanks}

Think tanks (e.g., Brookings, Heritage, AEI, CAP, Cato) play a major role in shaping public discourse about policy. They
often produce:

\begin{itemize}
    \item whitepapers,
    \item policy briefs,
    \item economic analyses,
    \item advocacy reports,
    \item commentary.
\end{itemize}

However, nearly all think tanks produce work shaped by underlying ideological commitments or donor priorities. As a result:

\begin{itemize}
    \item evaluations differ radically across institutions;
    \item frameworks for analysis vary widely;
    \item there is no unified or nonpartisan definition of policy quality;
    \item methodology is often opaque or narrative-driven;
    \item conclusions may be advocacy-oriented rather than diagnostic.
\end{itemize}

\paragraph{How PoliScore differs.}

PoliScore:

\begin{itemize}
    \item does not advocate for policy positions;
    \item uses a transparent, standardized framework and aggregation model;
    \item applies the same evaluative criteria to all legislation and legislators;
    \item grounds analysis in political philosophy, institutional economics, and governance theory rather than ideology;
    \item produces structured, replicable outputs, not narrative persuasion.
\end{itemize}

Think tanks answer: \textit{``Is this policy aligned with our values or goals?''}\\
PoliScore aims to answer: \textit{``How structurally sound is this policy, and what is the aggregate Intended Societal
Impact of a legislator's actions?''}

\subsection{Academic Public Policy and Political Science Programs}

Academic programs teach:

\begin{itemize}
    \item cost-benefit analysis,
    \item program evaluation,
    \item ethics and justice,
    \item public administration,
    \item implementation theory.
\end{itemize}

However:

\begin{itemize}
    \item methods vary by institution and instructor;
    \item frameworks are conceptual, not standardized;
    \item academics rarely evaluate policy before implementation at scale;
    \item analyses are usually qualitative, not rubric-based and automated;
    \item no unifying cross-disciplinary ``policy quality standard'' or legislator aggregation model exists.
\end{itemize}

\paragraph{How PoliScore differs.}

PoliScore:

\begin{itemize}
    \item operationalizes academic theory into a single coherent rubric;
    \item formalizes seven dimensions into measurable criteria;
    \item evaluates policy pre-implementation and at scale;
    \item integrates insights from economics, philosophy, governance, and systems engineering into a concrete scoring
    pipeline.
\end{itemize}

Academia answers: \textit{``How should we think about policy and governance?''}\\
PoliScore aims to answer: \textit{``How well-constructed is this specific policy, and what does a legislator's record look
like when evaluated under those standards?''}

\subsection{International Organizations and Independent Models}

International organizations and independent economic modeling groups evaluate:

\begin{itemize}
    \item development programs,
    \item governance indicators,
    \item effectiveness of existing policies,
    \item macroeconomic consequences.
\end{itemize}

Their analyses are valuable but limited in scope. They mainly:

\begin{itemize}
    \item evaluate implemented programs, not proposed legislation;
    \item rely on national data and long-term outcomes;
    \item focus on specific domains;
    \item analyze macro-level indicators rather than the structure of individual bills or the cumulative record of
    individual legislators.
\end{itemize}

PoliScore, if fully developed, would complement these efforts by:

\begin{itemize}
    \item evaluating legislation prospectively, before implementation;
    \item operating at the level of bill text and individual legislators;
    \item providing a cross-domain framework that can be applied uniformly;
    \item focusing on structural design quality and predicted societal impact rather than only observed outcomes.
\end{itemize}

\subsection{Summary of Differences}

Table~\ref{tab:institutions} summarizes the relationships between existing institutions and PoliScore.

\begin{table}[h]
    \centering
    \begin{tabular}{>{\raggedright\arraybackslash}p{0.19\textwidth}%
                    >{\raggedright\arraybackslash}p{0.23\textwidth}%
                    >{\raggedright\arraybackslash}p{0.23\textwidth}%
                    >{\raggedright\arraybackslash}p{0.27\textwidth}}
        \toprule
        \textbf{Institution Type} & \textbf{What They Evaluate} & \textbf{What They Do Not Evaluate} & \textbf{PoliScore’s Intended Contribution} \\
        \midrule
        CBO &
        Budgetary/fiscal impacts &
        Governance, feasibility, fairness, systemic risk, individual records &
        Provides holistic pre-implementation structural and impact evaluation, plus legislator aggregation via Intended Societal Impact. \\
        \midrule
        Think Tanks &
        Ideology-driven arguments &
        Standardized, nonpartisan evaluation &
        Supplies neutral, structured scoring across seven dimensions and a transparent bill-to-legislator model. \\
        \midrule
        Academia &
        Theory, ethics, evaluation methods &
        Unified applied rubric, automated scoring, legislator grades &
        Operationalizes theory into a consistent, scalable framework. \\
        \midrule
        International Orgs \& Macro Models &
        Retrospective outcomes; macro projections &
        Pre-implementation structural analysis of bills; individual legislator histories &
        Evaluates proposals before enactment and aggregates impacts to individual legislators. \\
        \bottomrule
    \end{tabular}
    \caption{Comparison of existing institutions and the envisioned role of PoliScore.}
    \label{tab:institutions}
\end{table}

%%%%%%%%%%%%%%%%%%%%%%%%%%%%%%%%%%%%%%%%%%%%%%%%%%%
%%%%%%%%%%%%%%%   LIMITATIONS   %%%%%%%%
%%%%%%%%%%%%%%%%%%%%%%%%%%%%%%%%%%%%%%%%%%%%%%%%%%%

\section{Limitations, Risks, and Ethical Considerations}
\label{sec:limitations}

While PoliScore and the PoliBench Benchmark Suite provide a structured and theoretically grounded framework for evaluating
public policy and legislative performance, they also introduce methodological, computational, and ethical challenges.
Recognizing these limitations is essential for responsible use and for ensuring that PoliScore complements, rather than
replaces, democratic decision-making and expert judgment.

In the current, unfunded phase of the project, many of these limitations are particularly acute: there is not yet a large
team to manage peer review, adversarial testing, or institutional partnerships. This section therefore serves both as a
warning label and as a roadmap for the governance structures that would be needed if PoliScore were adopted more broadly.

\subsection{Limitations of the Framework}

\subsubsection{Incomplete Representations of Policy Context}

Legislative text does not always contain:

\begin{itemize}
    \item administrative history,
    \item political constraints,
    \item agency capabilities,
    \item cultural factors,
    \item stakeholder incentives,
    \item implementation environment.
\end{itemize}

PoliScore evaluates text as written and explicit interaction data, not the full political or institutional context.
Real-world outcomes may differ from what the text and modeled impacts suggest.

\subsubsection{Dependence on Model Interpretation}

PoliScore currently relies on large language models for bill-level evaluation. While PoliBench is
intended to validate baseline competence, no model is infallible. AI systems may:

\begin{itemize}
    \item misinterpret ambiguous sections,
    \item overlook subtle governance issues,
    \item fail to identify complex incentive structures,
    \item inconsistently explain their reasoning,
    \item exhibit sensitivity to prompt phrasing.
\end{itemize}

Human oversight remains essential, particularly for high-impact legislation.

\subsubsection{Normative Judgments in Pillars and Aggregation}

Although the seven dimensions and aggregation rules are derived from widely accepted principles, any evaluative framework
contains implicit assumptions. For example:

\begin{itemize}
    \item principles of ``fairness'' rely on particular philosophical traditions;
    \item feasibility depends on assumptions about institutional capacity;
    \item sector weights and interaction weights embody normative views about what matters most.
\end{itemize}

PoliScore mitigates this by making all assumptions explicit and configurable, but it is not value-free.

\subsubsection{Challenges in Quantifying Qualitative Constructs}

Some aspects of policy quality---such as institutional trust, political legitimacy, or cultural acceptance---are inherently
difficult to quantify. PoliScore focuses on the structural and impact dimensions that can be reasoned about from text and
well-specified criteria, but cannot capture all nuances of real-world politics.

\subsubsection{Early-Stage Field}

Policy quality engineering is a new field. As such:

\begin{itemize}
    \item the framework will evolve,
    \item new dimensions or sub-dimensions may be added,
    \item weighting schemes may be refined,
    \item benchmark tasks will need continuous updates.
\end{itemize}

Versioning and transparent changelogs are therefore essential, especially as more collaborators become involved.

\subsection{Risks Associated With AI-Assisted Policy and Legislator Evaluation}

\subsubsection{Overreliance on AI Outputs}

AI-generated evaluations, if misunderstood as authoritative, may:

\begin{itemize}
    \item overshadow legitimate political debate;
    \item reduce the perceived role of experts and stakeholders;
    \item disincentivize democratic deliberation;
    \item be mistaken for objective truth rather than structured analysis.
\end{itemize}

PoliScore is designed as an analytical tool, not a source of political mandates.

\subsubsection{Risk of Misuse or Politicization}

Any scoring system can be misused or selectively weaponized. Risks include:

\begin{itemize}
    \item cherry-picking PoliScore results to support partisan narratives;
    \item misrepresenting composite scores without context;
    \item selectively highlighting favorable dimensions or interactions;
    \item using grades to target political opponents without acknowledging framework assumptions.
\end{itemize}

Mitigation requires full publication of methods, dimension-level breakdowns, and underlying bill-level scores, and ideally
some form of independent oversight or advisory board.

\subsubsection{Model Bias and Training Data Influence}

Even when the framework is neutral, AI models may incorporate biases from:

\begin{itemize}
    \item training data composition,
    \item institutional assumptions embedded in public discourse,
    \item coverage biases in scientific literature or media.
\end{itemize}

PoliScore's focus on ``Overall Impact to Society'' and evidence-based reasoning pushes models toward mainstream scientific
and public consensus on many topics, but this may not align with all ideological perspectives.

\subsubsection{Vulnerability to Adversarial Design}

Sophisticated actors could attempt to influence model outputs by:

\begin{itemize}
    \item strategically crafting bill text to appear more beneficial under known evaluation criteria;
    \item embedding misleading language that exploits LLM weaknesses;
    \item gaming sector-specific scoring patterns.
\end{itemize}

Ongoing research, adversarial testing, and human review are required to identify and mitigate such strategies.

\subsection{Ethical Considerations}

\subsubsection{Transparency and Explainability}

Users must understand:

\begin{itemize}
    \item how PoliScore generates evaluations,
    \item what each dimension and sector score means,
    \item how the aggregation into legislator grades is performed.
\end{itemize}

PoliScore therefore emphasizes:

\begin{itemize}
    \item open access to scoring rules and prompts,
    \item clear documentation of pipeline steps,
    \item availability of the underlying bill-level and interaction-level data used for aggregation.
\end{itemize}

\subsubsection{Human Oversight and Democratic Authority}

AI models and algorithmic scoring frameworks must not:

\begin{itemize}
    \item replace elected representatives;
    \item override democratic decision-making;
    \item be treated as infallible arbiters of political truth.
\end{itemize}

PoliScore is intended to help voters, researchers, and policymakers reason more clearly about policy and legislative
performance, not to dictate outcomes.

\subsubsection{Inclusivity in Framework Development}

As PoliScore evolves, its legitimacy depends on:

\begin{itemize}
    \item peer review,
    \item collaboration with academics and practitioners,
    \item multidisciplinary input from economics, law, public health, and other domains,
    \item feedback from civic organizations and the broader public.
\end{itemize}

An inclusive development process helps reduce blind spots and increases trust. At present, such a process does not yet
exist around PoliScore; building one will require support from institutions that care about non-partisan democratic
infrastructure.

\subsection*{Summary}

PoliScore introduces a rigorous, first-principles approach to evaluating policy quality and legislative performance, but it
is not a substitute for human judgment or democratic deliberation. Recognizing its limitations and potential risks is
essential to its responsible use. By emphasizing transparency, interpretability, cross-checking, and academic grounding,
PoliScore aims to serve as a constructive analytical tool rather than an authoritative arbiter of political outcomes. To
make that aspiration real, the project will need governance, funding, and stewardship that extend beyond its current,
single-author prototype status.

%%%%%%%%%%%%%%%%%%%%%%%%%%%%%%%%%%%%%%%%%%%%%%%%%%%
%%%%%%%%%%%%%%%   SECTION 7: WEB SEARCH   %%%%%%%%%
%%%%%%%%%%%%%%%%%%%%%%%%%%%%%%%%%%%%%%%%%%%%%%%%%%%

\section{Integrating Web Search: Benefits, Complexities, and Concerns}
\label{sec:web}

As AI systems become increasingly capable of retrieving and synthesizing information from the web, responsible integration
of external data into policy evaluation becomes both an opportunity and a challenge. Web search can significantly
strengthen PoliScore's analytical depth, especially for feasibility and evidence-based assessment, but it also introduces
complexities related to reliability, bias, reproducibility, and governance.

The considerations outlined in this section largely mirror those that apply to AI-assisted analysis in general, but with
additional emphasis on preserving the non-partisan and reproducible nature of PoliScore. In the current phase of the
project, these ideas are primarily design goals for future, well-resourced iterations.

\subsection{Benefits of Web Search Integration}

Web search provides access to real-world information that extends beyond the content of legislative text. When properly
constrained, it can:

\begin{itemize}
    \item supply up-to-date empirical data (e.g., workforce statistics, infrastructure capacity),
    \item clarify agency structures and institutional mandates,
    \item retrieve historical examples of similar policies,
    \item contextualize budget numbers and program scales.
\end{itemize}

These capabilities directly support:

\begin{itemize}
    \item Pillar 2 (Evidence Base \& Empirical Support),
    \item Pillar 3 (Implementation Feasibility),
    \item Pillar 4 (Economic Efficiency \& Fiscal Sustainability),
    \item Pillar 6 (Governance Integrity \& Institutional Risk).
\end{itemize}

Web search also helps reduce hallucination by anchoring model outputs in verifiable sources.

\subsection{Complexities and Risks}

However, integrating web search raises several challenges:

\begin{itemize}
    \item \textbf{Non-reproducibility.} Search results change over time and may vary by geography or search engine configuration.
    \item \textbf{Source quality.} The web contains a mix of official data, academic work, advocacy, and misinformation.
    \item \textbf{Ranking bias.} Search engine algorithms may implicitly prioritize certain narratives or domains.
    \item \textbf{Domain imbalance.} Some policy areas are richly documented, others are sparse.
\end{itemize}

Without strict guardrails, web search could erode the non-partisan foundation of PoliScore and undermine trust.

\subsection{Safeguards and Design Principles}

To mitigate these risks, any web-integrated variant of PoliScore should:

\begin{itemize}
    \item restrict retrieval to vetted source categories (e.g., official government domains, major statistical agencies,
    peer-reviewed journals, reputable international organizations),
    \item emphasize retrieval of \emph{facts} (e.g., numeric data, definitions) rather than opinion or advocacy,
    \item cache retrieved documents and associate them with timestamps and citations for reproducibility,
    \item expose retrieved sources in public reports so that users can independently evaluate them,
    \item flag evaluations that rely heavily on sparse or low-confidence external data.
\end{itemize}

PoliBench can be extended with retrieval-dependent tasks to test model behavior under these constraints.

%%%%%%%%%%%%%%%%%%%%%%%%%%%%%%%%%%%%%%%%%%%%%%%%%%%
%%%%%%%%%%%%%%%   SECTION 8: CONCLUSION   %%%%%%%%%
%%%%%%%%%%%%%%%%%%%%%%%%%%%%%%%%%%%%%%%%%%%%%%%%%%%

\section{Conclusion and Future Directions}
\label{sec:conclusion}

Public policy and legislative performance together define a large part of the lived experience of citizens. Yet until now,
there has been no unified, non-partisan, and methodologically rigorous framework for evaluating the structural quality of
legislation and aggregating its expected impacts into clear, interpretable assessments of what legislators have actually
done.

PoliScore addresses this gap by:

\begin{itemize}
    \item articulating a principled, interdisciplinary theory of policy quality grounded in human needs, political
    philosophy, institutional economics, governance theory, and systems thinking;
    \item formalizing this theory into a seven-pillar evaluation framework and a sectoral impact model that yield an
    ``Overall Impact to Society'' score for each bill;
    \item defining a transparent aggregation pipeline from bill-level impacts to legislator-level Intended Societal Impact
    scores and letter grades;
    \item providing parameterized natural-language summaries that help voters understand complex legislative histories;
    \item introducing PoliBench, a benchmark suite for measuring AI competence in policy reasoning and ensuring a baseline
    level of evaluator reliability.
\end{itemize}

Together, these components represent early steps in what may become a broader field: \textit{policy quality engineering}---
a discipline focused on the structural, empirical, and institutional soundness of public policy design and on principled
aggregation of those designs into measures of legislative performance.

\subsection{Opportunities for Future Work}

Several avenues for future research and development are evident:

\begin{itemize}
    \item \textbf{Empirical validation.} Applying PoliScore retrospectively to historical legislation and comparing
    predictions to observed outcomes can help calibrate weights, refine dimensions, and validate predictive power.

    \item \textbf{Collaboration with academia.} Partnerships with universities and research centers can stress-test the
    framework, refine benchmarks, and extend the theoretical foundations.

    \item \textbf{Sector-specific extensions.} Domain-specific sub-frameworks (e.g., for healthcare, climate policy, tax
    reform) can be developed atop the core pillars.

    \item \textbf{Richer retrieval and evidence systems.} Carefully constrained and documented use of web-based and
    curated data sources can strengthen evidence-based evaluation.

    \item \textbf{Internationalization.} Adapting the framework for use in other jurisdictions, legal traditions, and
    languages can broaden its applicability. Because the PoliScore implementation is released under the MIT license, such
    adaptations can be carried out by independent institutions that fork or extend the core codebase, reusing the shared
    definitions of policy quality while substituting local data feeds, institutional mappings, and sector weights. Over
    time, a constellation of jurisdiction-specific PoliScore deployments could emerge, each tuned to its own rule-of-law
    environment but grounded in a common evaluative vocabulary.

    \item \textbf{Civic integration.} Tools built on PoliScore could support voters, journalists, advocacy groups, and
    even legislators themselves in understanding and improving legislative performance.
\end{itemize}

\subsection{A Call for Collaboration and Support}

At the time of this writing, PoliScore is not backed by a foundation, a university, or a government grant. It is a
person-scale project trying to sketch what a more sane, quality-focused information layer over democratic institutions
might look like. As such, there are hard limits to how far it can go without help.

The vision outlined here---a world where voters can see through partisan narratives to the structural quality and
predicted impact of the laws being proposed in their name---will require:

\begin{itemize}
    \item \textbf{Engineers and data practitioners} to build and harden the pipelines, data models, and public interfaces.
    \item \textbf{Researchers and academics} to critique the framework, refine the pillars, and help construct PoliBench as
    a serious domain benchmark.
    \item \textbf{Civic technologists and journalists} to explore how PoliScore-style analysis can be integrated into
    reporting, voter guides, and accountability tools.
    \item \textbf{Institutional partners}---nonprofits, universities, or public-interest labs---willing to host and
    steward an open, non-partisan infrastructure.
    \item \textbf{Funders} who recognize that democratic health depends not just on voting procedures, but on the quality
    of information citizens have about the laws being written and the people writing them.
\end{itemize}

Because the codebase is MIT-licensed, participation can take many forms: contributing directly, forking and adapting,
building independent tools that plug into the same conceptual framework, or simply subjecting the ideas to serious,
public critique. The key is that the work moves us away from a politics dominated by memes and outrage, and toward a
shared concern with whether our laws are actually well-designed for human flourishing.

\subsection{Closing Remarks}

The challenges facing modern societies---economic transformation, climate change, technological disruption, public health,
and geopolitical instability---demand a new level of clarity and rigor in how we evaluate both policies and the people who
make them. PoliScore offers a concrete, transparent, and extensible starting point: a way to move discussions from partisan
narratives toward structured analysis rooted in human welfare and institutional robustness.

By framing both bills and legislators in terms of predicted societal impact, and by making every step of the pipeline
explicit and inspectable, PoliScore aims to give voters, researchers, and policymakers a clearer view of what our laws are
likely to do---and what our representatives have actually done. For now, it is a vision sustained by a single developer.
Whether it becomes a durable part of our democratic infrastructure will depend on whether others see value in that vision
and decide to help build it.

%%%%%%%%%%%%%%%%%%%%%%%%%%%%%%%%%%%%%%%%%%%%%%%%%%%
%%%%%%%%%%%%%%%   BIBLIOGRAPHY   %%%%%%%%%%%%%%%%%%
%%%%%%%%%%%%%%%%%%%%%%%%%%%%%%%%%%%%%%%%%%%%%%%%%%%

\newpage
\begin{thebibliography}{9}

\bibitem{rawls1971}
John Rawls,
\textit{A Theory of Justice}.
Harvard University Press, 1971.

\bibitem{sen1999}
Amartya Sen,
\textit{Development as Freedom}.
Oxford University Press, 1999.

\bibitem{ostrom1990}
Elinor Ostrom,
\textit{Governing the Commons}.
Cambridge University Press, 1990.

% Additional references to be added in future revisions as the framework and benchmark are further developed.

\end{thebibliography}

\end{document}
